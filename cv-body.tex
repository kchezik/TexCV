\section{Education}

\begin{description}
\tightlist
\item[2013--(\textit{Exp.}19)]
\textbf{Ph.D.}\ Simon Fraser University, Burnaby BC, Canada\\
Department of Biological Sciences\\
\textit{Thesis title}: River network structuring of climate and landscape effects in salmon watersheds.\\
\textit{Advisor}: Dr. Jonathan Moore\\

\item[2011--13]
\textbf{M.Sc.}\ University of Minnesota, St. Paul MN, USA\\
Department of Fisheries, Wildlife and Conservation Biology\\
\textit{Thesis title}: Fish growth and degree-days: Advice for selecting base temperatures in both within- and among- lake studies.\\
\textit{Advisor}: Dr. Paul Venturelli\\

\item[2005--09]
\textbf{B.A.}\ St. Olaf College, Northfield MN, USA\\
\textit{Major/Concentration}: Biology/Environmental Studies\\
\textit{Advisors}: Dr. John Schade \& Dr. Patrick Ceas\\
\end{description}



\section{Personal statement}

I am interested in the impacts of human activities on systems and using data driven methods to understand complex interactions and predict potential outcomes of management with the aim of facilitating knowledge driven decisions in the face of uncertainty. My graduate work has focused primarily on the impacts of climate change on physical and biological processes, but I’ve also been interested in developing methods to mitigate bias in statistical models and automate data quality control through unsupervised machine learning. For instance, some of my work described how the arborescent structure of river networks dampens climate driven shifts in river flow by aggregating a diverse climate portfolio. I’ve also described how the spatial structure of climate, conveyed to the river network, has varied impacts on salmon populations inhabiting different locations on the network. These research questions required large volumes of data, advanced statistical techniques and technical knowledge of computing software, hardware and languages. Process challenges associated with research have made me keenly interested in developing tools that reduce the labor associated with ‘big data’, leading me to build a Hidden Morkov Model, an unsupervised machine learning method, to probabilistically identify errors in stream temperature data. Ultimately, my passion is uncovering patterns in data and displaying these relationships in clever ways that are beautiful and intuitive, in order to convey knowledge in a digestible way that can reach those in positions to make important decisions.



\section{Technical skills}

\begin{description}
\tightlist
\item[Stats] Generally proficient at frequentist, maximum likelihood and Bayesian methodologies, with specific experience building Hidden Markov Models, Linear Regressions, Generalized Additive Models, Mixed Effect Models, ARMA, ARIMA and GARCH models, State Space Models, Spatial/Temporal Autocorrelation models, $\dots$\\
\item[Langs] R, Python, Mac Linux Command Line, Stan, LaTex, HTML, CSS, C, Regex, (by proficiency)\\
\item[DB Mgt] Relational – (SQL, MS Access, …) Conventional – (Excel, CSV, RDS, $\dots$)\\
\item[GIS] ArcGIS, QGIS, SAGA, Whitebox, OSGEO, GDAL, $\dots$\\
\item[Writing] Proposals, technical reports, manuscripts, $\dots$\\

\end{description}



\section{Professional experience}

\begin{description}
\tightlist

\item[2013--19] Graduate Research Assistant, Department of Biological Sciences, Simon Fraser University, Burnaby, BC, Canada
\item[2011--13] Graduate Research Assistant, Department of Fisheries, Wildlife and Conservation Biology, University of Minnesota, St. Paul, MN, USA
\item[2011] Associate Medical Device Reporting Specialist (MDR), Medtronic Inc. Mounds View, MN, USA
\item[2009--10] Data Specialist/CSSC II, Kelly Scientific Services, St. Louis Park, MN, USA
\item[2008-09] Field Research Assistant, St. Olaf Collaborative Undergraduate Research and Inquiry (CURI) program, Northfield, MN, USA

\subsection{Contract employment}

\item[2016] Database development and analysis for the Pacific Salmon Foundation under sub-contract with ESSA Inc., towards a working stream network temperature model for British Columbia

\end{description}



\section{Publications}

\begin{description}
\tightlist

\item[\textit{In prep}] Chezik, K.A., Moore, J.W. Impacts of network structure and landscape complexity on the cumulative thermal exposure of migrating salmon. Target journal: TBD
\item[\textit{In prep}] Chezik, K.A. Cleaning stream temperature data with hidden markov models. Target journal: Methods in Ecology.
\item[\textit{In prep}] Chezik, K.A., Wilson, S.M., Moore, J.W. Spatial structuring of climate match-mismatch in a migratory fish. Target Journal: Proceedings of the Royal Society B.
\item[2017] Chezik, K.A., Anderson, S.C., Moore, J.W. River networks dampen long-term hydrological signals of climate change. Geophysical Research Letters 44: 7256-726 \url{https://doi-org.proxy.lib.sfu.ca/10.1002/2017GL074376}.
\item[2014] Chezik, K.A., Nigel L.P., Venturelli, P.A. Fish growth and degree-days II: Selecting a base temperature for an among-population study. Canadian Journal of Fisheries and Aquatic Sciences 71(1): 1303-1311 \url{https://doi-org.proxy.lib.sfu.ca/10.1139/cjfas-2013-0615}.
\item[2014] Chezik, K.A., Nigel L.P., Venturelli, P.A. Fish growth and degree-days I: Selecting a base temperature for a within-population study. Canadian Journal of Fisheries and Aquatic Sciences 71(1): 47-55 \url{https://doi-org.proxy.lib.sfu.ca/10.1139/cjfas-2013-0295}.

\end{description}



\section{Conference presentations}

\begin{description}
\tightlist

\item[2018] Chezik, K.A., Wilson, S.W., Moore, J.W. Spatial patterns of phenological match-mismatch in pink salmon. Association for the Sciences of Limnology and Oceanography. Victoria, BC. (Oral)
\item[2017] Chezik, K.A., Anderson, S.C., Moore, J.W. River networks dampen long-term hydrological signals of climate change. Canadian Society of Ecology and Evolution. Victoria, BC. (Oral)
\item[2017] Chezik, K.A., Anderson, S.C., Moore, J.W. River networks dampen long-term hydrological signals of climate change. American Water Resources Association. Salt Lake City, UT. (Invited-Oral)
\item[2015] Chezik, K.A., Anderson, S.C., Moore, J.W. River networks: river networks attenuate climate-induced flow trends. American Fisheries Society. Portland, OR. (Oral)
\item[2013] Chezik, K.A., Nigel L.P., Venturelli, P.A. The first steps towards a standardized approach to using degree-days in fish science. Ecological Society of America. Minneapolis Minnesota. (Oral)
\item[2013] Chezik, K.A., Nigel L.P., Venturelli, P.A. Degree-days in fish science: an argument for the standardization of base temperatures. Symposium for European Freshwater Sciences. Münster Germany. (Oral)
\item[2012] Chezik, K.A., Nigel L.P., Venturelli, P.A. Degree-day thresholds: towards a standardized approach to using degree-days in fish science. American Fisheries Society. St. Paul MN. (Oral)
\item[2012] Chezik, K.A. Nigel L.P., Venturelli, P.A. 2012. Degree-Day Thresholds: Towards a standardized approach to using degree-days in fish science. International Congress on the Biology of Fish. Madison, WI. (Oral)

\end{description}



\section{Teaching \& workshops}

\begin{description}
\tightlist

\item[2016] \textit{Skeena Fisheries Commission: Introduction to R workshop}
\item[2013] \textit{Analysis of populations} teaching assistant
\item[2011] \textit{Fisheries population analysis} teaching assistant

\end{description}



\section{Funding \& awards}

\begin{description}
\tightlist

\item[2018] KEY Big Data Graduate Scholarship, Simon Fraser University 
\item[2015--18] Travel \& Minor Research Award, Simon Fraser University
\item[2016--17] Graduate Fellowship, Simon Fraser University
\item[2013] Travel Grant, University of Minnesota
\item[2013] Fisheries Society of the British Isles Travel Grant
\item[2013] Block Grant, University of Minnesota 
\item[2009] Behrent's Research Grant, St. Olaf College 
\item[2009] Volunteer Network Program of the Year Award , St. Olaf College

\end{description}


\section{References}

Dr. Jonathan W. Moore, Associate Professor\\
Liber Ero Chair of Aquatic Ecology and Conservation\\
Simon Fraser University, Department of Biological Sciences\\
Phone: (778) 782-9246,	Email: jwmoore@sfu.ca\\

Dr. Sean C. Anderson, Senior Aquatic Science Biologist\\
Fisheries and Oceans Canada\\
Pacific Biological Station, Nanaimo BC Canada\\
Phone: (250) 756-7171,	Email: sean.anderson@df-mpo.gc.ca\\

Dr. Paul A. Venturelli, Assistant Professor of Fisheries\\
Ball State, Department of Biology\\
Phone: (765) 285-8812,	Email: paventurelli@bsu.edu\\

Additional references available upon request.
